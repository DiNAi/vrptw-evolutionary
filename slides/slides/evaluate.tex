\begin{frame}
        \begin{block}{Funkcja celu}
                Jeśli przez $t_i$ oznaczymy czas dotarcia do wierzchołka $v_i$ to rozwiązanie możemy ocenić za pomocą funkcji celu której interpretacja to suma kar jakie ponieśliśmy:
\begin{equation}
        f = \sum\limits_{i=2}^{n} m_i
\end{equation}
\begin{equation}
m_i = \text{if} \ t_i > d_i \ \text{then} \ k_i \ \text{else} \ 0
\end{equation}
        \end{block}
\end{frame}

\begin{frame}
Warianty problemu:
\begin{itemize}
\item ograniczenia na \textbf{limit pojemności} pojazdu wymuszający odwiedzanie magazynu,
\item \textbf{okna dostawy} (najpóźniejszy jak i najwcześniejszy czas dostawy),
\item minimalizacja \textbf{liczby błędów} zamiast minimalizacji funkcji kary.
\end{itemize}
\begin{block}{Ocena}
    W drugim przypadku oceną rozwiązania jest krotka (liczba błędów, długość przejazdu) i wyniki sortujemy leksykograficznie, tj. w pierwszej kolejności minimalizujemy liczbę przekroczonych terminów, a dopiero potem długość trasy.
\end{block}

\end{frame}
