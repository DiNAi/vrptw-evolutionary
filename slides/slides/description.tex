\begin{frame}
	\begin{block}{Opis problemu (orginalny)}
		Mamy dany ważony graf pełny miast z magazynem (punktem startowym). Wagi na krawędziach interpretujemy jako czasy potrzebne na przemieszczenie sie między miastami. Dodatkowo każdy wierzchołek ma przypisany najpóźniejszy czas dostawy za przekroczenie którego otrzymujemy karę. 
		Ciężarówka ma nieskończoną pojemność wiec raz załadowana może jeździć dowolnie długo.
		Znamy również maksymalny czas podróży ciężarówki po jakim żadne zadanie nie jest dopuszczalne.
		Należny znaleść ciag miast, taki że każde miasto jest odwiedzone i suma kar jest minimalna.
	\end{block}
\end{frame}

\begin{frame}
	\begin{block}{Instancja problemu}
		Miastom przyporządkowujmey kolejno numery $2, \cdots, n$.
		Dla uproszczenia będziemy zakładać, że magazyn zawsze ma numer $1$.
		Instancją problemu jest trójka:
		\begin{equation}
			P = \left< T, K, D\right>
		\end{equation}
	\end{block}
	\begin{block}{Macierz czasu przejazdów}		
		\begin{equation}
			T = \left[ t_{ij} \right]_{n \times n} 
		\end{equation}
		Kwadratowa maciarz gdzie element $t_{ij}$ to czas potrzebny na przejazd z miasta $i$ do $j$.
	\end{block}
	\begin{block}{K i D}
		Wektory K i D przyporządkowują wierzchołkom odpowiednio najpóźniejsze czasy dostawy oraz karę za ich przekroczenie.
	\end{block}
\end{frame}

\begin{frame}
	\begin{block}{Rozwiązania dopuszczalne}
		Rozwiązaniem dopuszczalnym (spełniającym warunki zadania) jest permutacja liczb $1, \cdots, n$.
	\end{block}

	\begin{block}{Uzasadnienie}
		\begin{itemize}
			\item W rozwiązniu musza znaleść sie wszystkie miasta. (definicja problemu)
			\item W rozwiązaniu miasta nie mogą sie powtarzać. (nierówność trójkąta
				+ założenie o nieskończonej ładowności ciężarówki)
			\item W rozwiązaniu magazyn musi być pierwszym wierzchołkiem.
		\end{itemize}
	\end{block}
\end{frame}

