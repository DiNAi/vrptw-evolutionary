\begin{block}{Nierówność trójkąta}
	\begin{equation}
		a < b + c \ \wedge \ b < a + c \ \wedge \ c < a + b
	\end{equation}
	\begin{itemize}
		\item Zakładamy, że dany czas przejazdu między dowolnymi dwoma miastami to średni czas potrzebny na pokonanie najszybszej trasy łączącej te dwa miasta.
		\item Dzięki temu założeniu graf dla miast spełnia nierówność trójkąta czyli:
			\begin{equation}
				\forall_{a,b,c \in V} t(\left\{a, c\right\}) < t(\left\{a, b\right\}) + t(\left\{b, c\right\})
			\end{equation}
		\item Eleminujemy możliwość rozwiązania gdzie miasta mogą sie powtarzać, jeżeli mamy dostarczyć towar do miasta $b$, a znajdujemy sie w mieścia $a$ to najszybsza droga między tymi miastami zajmuje $t(\left\{a, b\right\})$.
	\end{itemize}

	
\end{block}
