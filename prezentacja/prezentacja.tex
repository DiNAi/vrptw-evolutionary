\documentclass{beamer}

\usepackage[polish]{babel}
\usepackage[T1]{fontenc}
\usepackage[utf8]{inputenc}
\usepackage{txfonts}
\usepackage{amsmath}
\usepackage{algorithm2e}

\usetheme{Warsaw}

\title{Problem  dystrybucji towarów z najpóźniejszymi terminami dostaw} 
\author{Piotr Rzepecki, Krzystof Zielonka, Krzysztof Dąbrowski}

\date{24 październik 2012}

\begin{document}

\frame{\titlepage}

\begin{frame}
	\begin{block}{Opis problemu (orginalny)}
		Mamy dany ważony graf pełny miast z magazynem (punktem startowym). Wagi na krawędziach interpretujemy jako czasy potrzebne na przemieszczenie sie między miastami. Dodatkowo każdy wierzchołek ma przypisany najpóźniejszy czas dostawy za przekroczenie którego otrzymujemy karę. 
		Ciężarówka ma nieskończoną pojemność wiec raz załadowana może jeździć dowolnie długo.
		Znamy również maksymalny czas podróży ciężarówki po jakim żadne zadanie nie jest dopuszczalne.
		Należny znaleść ciag miast, taki że każde miasto jest odwiedzone i suma kar jest minimalna.
	\end{block}
\end{frame}

\begin{frame}
	\begin{block}{Instancja problemu}
		Miastom przyporządkowujmey kolejno numery $2, \cdots, n$.
		Dla uproszczenia będziemy zakładać, że magazyn zawsze ma numer $1$.
		Instancją problemu jest trójka:
		\begin{equation}
			P = \left< T, K, D\right>
		\end{equation}
	\end{block}
	\begin{block}{Macierz czasu przejazdów}		
		\begin{equation}
			T = \left[ t_{ij} \right]_{n \times n} 
		\end{equation}
		Kwadratowa maciarz gdzie element $t_{ij}$ to czas potrzebny na przejazd z miasta $i$ do $j$.
	\end{block}
	\begin{block}{K i D}
		Wektory K i D przyporządkowują wierzchołkom odpowiednio najpóźniejsze czasy dostawy oraz karę za ich przekroczenie.
	\end{block}
\end{frame}

\begin{frame}
	\begin{block}{Rozwiązania dopuszczalne}
		Rozwiązaniem dopuszczalnym (spełniającym warunki zadania) jest permutacja liczb $1, \cdots, n$.
	\end{block}

	\begin{block}{Uzasadnienie}
		\begin{itemize}
			\item W rozwiązniu musza znaleść sie wszystkie miasta. (definicja problemu)
			\item W rozwiązaniu miasta nie mogą sie powtarzać. (nierówność trójkąta
				+ założenie o nieskończonej ładowności ciężarówki)
			\item W rozwiązaniu magazyn musi być pierwszym wierzchołkiem.
		\end{itemize}
	\end{block}
\end{frame}



\begin{frame}
	\begin{block}{Funkcja celu}
		Jeśli przez $t_i$ oznaczymy czas dotarcia do wierzchołka $v_i$ to rozwiązanie możemy ocenić za pomocą funkcji celu której interpretacja to suma kar jakie ponieśliśmy:
\begin{equation}
	f = \sum\limits_{i=2}^{n} m_i
\end{equation}
\begin{equation}
m_i = \text{if} \ t_i > d_i \ \text{then} \ k_i \ \text{else} \ 0
\end{equation}
	\end{block}


\end{frame}


%\begin{frame}{Złożonośc problemu}
	\begin{block}{Złożnosc problemu}
		Problem dystrupucji towarów jest NP trudny.
		Udowodnimy to konstrująć wielomianową redukcje problemu komiwojażera do problemu dystrybucji towarów.
	\end{block}

	\begin{block}{Problem komiwojażera}
		Mamy dany pełny ważony graf G. Rozwiązaniem problemu jest minimalny cykl Hamiltona na tym grafie.
	\end{block}
\end{frame}

\begin{frame}{Złożonośc problemu}
	\begin{block}{Dowód}
		
		\begin{algorithm}[H]
			\KwData{G -graf pełny wazony, w - funkcja wagowa}
			\KwResult{trojka $\left<T, p, t_{max}\right>$}
			$f$ -- funkcja przyporządkowywujaca wierzchołkom grafu kolejne liczby naturalne \\
			$g$ -- funkcja przyporządkowwywująca krawędziom kolejne liczby naturalne \\
			$p(v, t) = \lambda\left( v, t \right) \rightarrow \sum_{\{u, v\} \in E} (2^{g(\left\{u, v\right\})} \& t) \cdot w(\left\{ u, v \right\}) $ \\
			\ForAll{$v, u \in V$}{
				$T[f(v), f(u)] = 2^{g(\left\{v,u\right\})}$ \\
			}
			$t_{max} = \sum\limits_{i=2}^{|V|} w(f(v_{i-1}), f(v_i))$ \\
			\Return $\left< T, p , t_{max} \right> $
		\end{algorithm}
	
	\end{block}

\end{frame}


%\begin{frame}{Algorytm konstrukcyjny}
	
	\begin{block}{Opis algorytmu konstrukcyjnego}
		Algorytm konstrukcyjny oparliśmy na podstawie algorytmu Local Search. 
		Ssąsiądów wybieramy wykorzystując wszystkie możliwe transpozycje.
		
	\end{block}

	\begin{block}{Local Search}
		\begin{algorithm}[H]
			\KwData{$x$ -- initial node}
			
			\KwResult{$best\_neighbour$ -- the best local node}
			$current = none$ \\
			$best\_neighbor = x$ \\
			\Repeat{$F(best_neighbour) = = F(current)$}{
				$current = best\_neighbour$
				$neighbours = $ find all neighbours of $current$
				$best\_neighbour = $ select best from $neighbours \cup \left\{current\right\}$				
			}
			\Return $best\_neighbour$
		\end{algorithm}
	\end{block}

\end{frame}

\begin{frame}{Algorytm konstrukcyjny}

	\begin{block}{Neighbors}
		\begin{algorithm}[H]
			\KwData{x -- current node}
			\KwResult{neighbours -- set of x's neighbours}
			$ neighbours = \empty $ \\
			\ForAll{$1 \leq i, j \leq n \ \wedge i < j$}{
				$neighbours \cup= x_1 \cdots x_{i-1}, x_j, x_{i+1} \cdots x_{j-1}, x_i, x_{j+1} \cdots x_n$
			}
			\Return neighbours
		\end{algorithm}
	\end{block}

\end{frame}

\begin{frame}{Algorytm konstrukcyjny}

	\begin{block}{Algorytm konstrykcyjny dla problemu dystrybucji towarów}

		\begin{algorithm}[H]
			\KwData{$T, p, t_{max}, x$ -- initial node}
			\KwResult{$best\_neighbor$ -- best local neighbour}
			$ current = none $ \\
			$ best\_neighbor = x $ \\
			\Repeat{$F(best_neighbour) = = F(current)$}{
				$current = best\_neighbor$
				$neighbours = Neighbours(x) \cup \left\{current\right\}$
				$best\_neighbour =  $ select best from $neighbours$
			}
			\Return $best\_neighbor$
		\end{algorithm}
	\end{block}

\end{frame}


%\begin{frame}{Problem  dystrybucji towarów z najwcześniejszymi i najpóźniejszymi terminami dostaw}
%	\begin{block}{Opis problemu}
\begin{itemize}
	\item Pełny graf ważony z $n$ wierzchołkami,
	\item Wyróżniony jeden wierzchołek startowy $v_{start}$
	\item Wprowadzmy funckje kary $p$ jaką trzeba zapłacić za dostarczenie towaru w czasie $t$ do pewnego miasta (bardziej ogólny wariant, miasta mogą nadawać kary bardziej swobodnie oraz mogą nadawać nagorody)
	\item Każda krawędź ma przyparzadkowany czas potrzebny na jej pokonanie.
	\item Dodatkowo zakładamy, że w każdym mieście możemy przeczekać pewien okres czasu.
	\item Dla uproszczenie zakładamy, że jest pewne górne ograniczenie na czas potrzebny na pokonaniem trasy. Jeżeli rozwiązanie potrzebuje więcej czasu zakładamy, że jest ono nieakceptowalne.
\end{itemize}
\end{block}

%\end{frame}

%\begin{frame}{Problem dystrybucji towarów z najwcześniejszymi i najpóźniejszymi terminami dostaw}
%	\begin{block}{Opis problemu}
	Przez $DT$ oznaczmy problem dystrybucji towarów:
	\begin{equation}
		DT = \left< V, w, p, t_{max}\right>
	\end{equation}
	%Graf:
	%\begin{equation}
	%	G = \left< V, E \right>
	%\end{equation}
	%\begin{equation}
	%	E = \left\{ \{v, u\} \ : \ v, u \in V \right\}
	%\end{equation}
	\begin{equation}
		t : V \rightarrow N
	\end{equation}
	\begin{equation}
		p : V \times N \rightarrow R
	\end{equation}
	\begin{description}
		\item[$w$] -- przyporządkowuje krawędziom wagi (czasy podróży)
		\item[$p$] -- funckja kary, dla danego wierzchołka i czasu przybcia zwraca kare w postaci liczby rzeczywistej
		\item[$t_{max}$] -- górne ograniczenie na czas potrzebny na pokonaniem dystansu
	\end{description}
\end{block}

%\end{frame}

%\begin{frame}
%	\begin{block}{Nierówność trójkąta}
	\begin{equation}
		a < b + c \ \wedge \ b < a + c \ \wedge \ c < a + b
	\end{equation}
	\begin{itemize}
		\item Zakładamy, że dany czas przejazdu między dowolnymi dwoma miastami to średni czas potrzebny na pokonanie najszybszej trasy łączącej te dwa miasta.
		\item Dzięki temu założeniu graf dla miast spełnia nierówność trójkąta czyli:
			\begin{equation}
				\forall_{a,b,c \in V} t(\left\{a, c\right\}) < t(\left\{a, b\right\}) + t(\left\{b, c\right\})
			\end{equation}
		\item Eleminujemy możliwość rozwiązania gdzie miasta mogą sie powtarzać, jeżeli mamy dostarczyć towar do miasta $b$, a znajdujemy sie w mieścia $a$ to najszybsza droga między tymi miastami zajmuje $t(\left\{a, b\right\})$.
	\end{itemize}

	
\end{block}

%\end{frame}

%\begin{frame}{Problem  dystrybucji towarów z najwcześniejszymi i najpóźniejszymi terminami dostaw}
%	\begin{block}{Cel}
\begin{itemize}
\item \textbf{Celem} jest znalezienie ścieżki startującej w $x$, która minimalizuje sumę wartości funkcji $F$ i $G$ oraz długość ścieżki,
\pause
\item Problem dystrybucji towarów z najwcześniejszymi i najpóźninejszymi terminami dostaw redukuje się do NP-zupełnego "Problemu Podziału na Podzbiory" \ [ang. \textit{SPP - Set Partitioning Problem}],
\end{itemize}
\end{block}

%\end{frame}

%\begin{frame}
%	\begin{block}{Rozwiązanie}
	* Rozwiązanie zawiera wszystkie wierzchołki.
	** Rozwiązanie nie zawiera cykli:
	\begin{itemize}
		\item Krawędzie spełniają nierówność trójkąta
		\item Przed rozładunkiem może przeczekać w danym mieście dowolny okres czasu nie płacąc żadnej kary
	\end{itemize}
	Rozwiązaniem jest permutacja wierzchołków z $V$.
\end{block}

%\end{frame}

%\begin{frame}
%	%\input{./slajdy/czas_przejazdu.tex}
%\end{frame}

%\begin{frame}{Problem  dystrybucji towarów z najwcześniejszymi i najpóźniejszymi terminami dostaw}
%	\begin{frame}
	\begin{block}{Funkcja celu}
		Jeśli przez $t_i$ oznaczymy czas dotarcia do wierzchołka $v_i$ to rozwiązanie możemy ocenić za pomocą funkcji celu której interpretacja to suma kar jakie ponieśliśmy:
\begin{equation}
	f = \sum\limits_{i=2}^{n} m_i
\end{equation}
\begin{equation}
m_i = \text{if} \ t_i > d_i \ \text{then} \ k_i \ \text{else} \ 0
\end{equation}
	\end{block}


\end{frame}

%\end{frame}

%\begin{frame}{Problem  dystrybucji towarów z najwcześniejszymi i najpóźniejszymi terminami dostaw}
%	 \begin{block}{Przestrzeń poszukiwań}
\begin{itemize}
\item W celu znalezienia rozwiązania instancji problemu dystrybucji towarów z najwcześniejszymi i najpóźninejszymi terminami dostaw, musimy rozważać zbiory wszystkich możliwych scieżek zaczynających się w $x$, w pełnym grafie.
\pause
\item Dla grafu n wierzchołkowego mamy 
\pause
\item  tutaj jebnać trzeba wzór
\end{itemize}
\end{block}

%\end{frame}

\end{document}
